\documentclass{article}
\usepackage{amsmath}
\usepackage{graphicx}
\usepackage{hyperref}

\title{Sample \LaTeX{} Document}
\author{Your Name}
\date{\today}

\begin{document}

\maketitle

\section{Introduction}
This is a sample \LaTeX{} document that demonstrates various features and formatting options.

\section{Mathematical Equations}
Here are some examples of mathematical equations:

Inline equation: $E = mc^2$

Display equation:
\begin{equation}
    \int_{a}^{b} f(x) \, dx = F(b) - F(a)
\end{equation}

Matrix example:
\begin{equation}
    \begin{pmatrix}
        a & b \\
        c & d
    \end{pmatrix}
\end{equation}

\section{Tables}
Here's a sample table:

\begin{table}[h]
    \centering
    \begin{tabular}{|c|c|c|}
        \hline
        Item & Quantity & Price \\
        \hline
        Apples & 5 & \$2.50 \\
        Oranges & 3 & \$1.80 \\
        Bananas & 6 & \$3.00 \\
        \hline
    \end{tabular}
    \caption{Sample Fruit Prices}
    \label{tab:fruits}
\end{table}

\section{Lists}
\subsection{Bullet Points}
\begin{itemize}
    \item First item
    \item Second item
    \item Third item
\end{itemize}

\subsection{Numbered List}
\begin{enumerate}
    \item First step
    \item Second step
    \item Third step
\end{enumerate}

\section{References}
You can reference the table above as Table~\ref{tab:fruits}.

\end{document}
